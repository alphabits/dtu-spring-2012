\documentclass[11pt]{article}
\linespread{1}

\renewcommand{\thefootnote}{\fnsymbol{footnote}}

\usepackage{geometry} % see geometry.pdf on how to lay out the page. There's lots.
\usepackage[utf8]{inputenc}
\usepackage[T1]{fontenc}
\usepackage{array}
\usepackage{amsmath,amssymb,latexsym,epic,eepic,epsfig,graphics,psfrag}
\usepackage{amsfonts}
\usepackage{graphicx,float}
\usepackage{color}
\usepackage{siunitx}
\sisetup{expproduct = \cdot}
\definecolor{mygray}{RGB}{244,244,244}
\definecolor{gray}{gray}{0.5}
\definecolor{myredish}{RGB}{193,33,97}
\definecolor{grayblue}{RGB}{91,112,142}
\definecolor{myorange}{RGB}{255,134,0}
\definecolor{green}{rgb}{0,0.4,0}

\usepackage[english]{babel}

\usepackage[bottom]{footmisc}

\usepackage{fancyhdr}
\pagestyle{fancy}
\lhead{\small\textit{Assignment \assignmentnumber -- 02433 Hidden Markov Models -- Anders Hørsted (s082382)}}
\rhead{\thepage}
\chead{}
\lfoot{}\cfoot{}\rfoot{}

\usepackage{subfigure}
\usepackage{placeins}
\usepackage{pstricks}
\usepackage{pst-node}
\usepackage{wrapfig}
\usepackage{multirow}
%\usepackage{fouriernc}
%\usepackage[charter]{mathdesign}
\usepackage{lmodern}
\usepackage[normalem]{ulem}
\geometry{a4paper} % or letter or a5paper or ... etc
% \geometry{landscape} % rotated page geometry

\usepackage{natbib}
\usepackage[font={it,small}]{caption}

\usepackage{url}


\makeatletter
\renewcommand*\env@matrix[1][*\c@MaxMatrixCols c]{%
  \hskip -\arraycolsep
  \let\@ifnextchar\new@ifnextchar
  \array{#1}}
\makeatother

\newcommand\myimp{\quad\Leftrightarrow\quad}
\newcommand\half{\frac{1}{2}}
\newcommand\myvec[1]{\boldsymbol{#1}}
\newcommand\mymod[1]{\ (\text{mod }#1)}
\newcommand\myreal{\mathbb{R}}
\newcommand\mynatural{\mathbb{N}}
\newcommand\myinteger{\mathbb{Z}}
\newcommand\mycomplex{\mathbb{C}}
\newcommand\myint{\text{int}}
\newcommand\norm[1]{||\,#1\,||}
\newcommand\bignorm[1]{\big|\big|\,#1\,\big|\big|}
\newcommand\seq[1]{\big\{#1\big\}}
\newcommand\smallseq[1]{\{#1\}}
\newcommand\smallseqtoinf[1]{\smallseq{#1}_{k=1}^\infty}
\newcommand\lonew{\ell^1_w}
\newcommand\lone{\ell^1}
\newcommand\ltwo{\ell^2(\mynatural)}
\newcommand\ip[2]{\langle#1,#2\rangle}
\newcommand\hilbert[1]{\mathcal{#1}}
\newcommand\uinf{u_{\infty}}
\newcommand\erf{\text{erf\,}}
\newcommand\infint{\int_{\infty}^{\infty}}
\newcommand\E[1]{\text{E}[#1]}
\newcommand\Var[1]{\text{Var}[#1]}
\newcommand\Cov[1]{\text{Cov}[#1]}
\newcommand\Corr[1]{\text{Corr}[#1]}
%\newcommand\Pr{\text{Pr}}
\newcommand\myverb[1]{{\footnotesize\texttt{#1}}}
\newcommand\Yhat{\widehat{Y}}
\newcommand\given{\,|\,}
\newcommand\acf{ACF}
\newcommand\pacf{PACF}
\newcommand\appref[1]{appendix~\ref{app:#1}}
\newcommand\coderef[1]{code listing~\ref{src:#1}}
\newcommand\todo[1]{\textbf{\textcolor{red}{#1}}}
\newcommand\plotpath[1]{../plots/#1}
\newcommand\respath[1]{../results/#1}
\newcommand\srcpath[1]{../src/#1}

\usepackage[scaled]{beramono}
\usepackage{listings}
\lstset {                 % A rudimentary config that shows off some features.
    language=R,
    basicstyle=\scriptsize\ttfamily, % Without beramono, we'd get cmtt, the teletype font.
    commentstyle=\textit, % cmtt doesn't do italics. It might do slanted text though.
    keywordstyle=,
    identifierstyle=,
    aboveskip=12pt,
    abovecaptionskip=6pt,
    framextopmargin=4pt,
    framexbottommargin=4pt,
    framexleftmargin=4pt,
    framexrightmargin=4pt,
    xleftmargin=4pt,
    xrightmargin=4pt,
    backgroundcolor=\color{mygray},
    frame=single,
    showstringspaces=false,
    captionpos=b,
    tabsize=4            % Or whatever you use in your editor, I suppose.
}

\renewcommand{\lstlistlistingname}{Code Listings}
\renewcommand{\lstlistingname}{Code Listing}

\usepackage{tabulary}
\newcolumntype{y}{>{\centering\arraybackslash}R}

\title{\assignmenttitle}
\date{\assignmentdate}
\author{Assignment \assignmentnumber\ -- 02433 Hidden Markov Models -- Anders Hørsted (s082382)}
%\author{}
\date{} % delete this line to display the current date

